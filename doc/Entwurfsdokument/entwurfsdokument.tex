\documentclass{article}
\usepackage[ngerman]{babel}
\usepackage{setspace}
\usepackage[utf8]{inputenc}
\usepackage{textcomp}
\usepackage{hyperref}

%für endliche Automaten und sonstige Zeichnungen
%\usepackage{tikz}
%\usetikzlibrary[arrows, automata]

\parindent0cm
\onehalfspacing

\usepackage{scrpage2}
\pagestyle{scrheadings}
\clearscrheadfoot
\ihead{A. Kern, A. Möller}
\chead{Big Data Praktikum}
\ohead{\today}

\title{Analytics of Publication Data with Graphulo - Entwurfsdokument}
\author{Alexander Kern \and Alexander Möller}

\begin{document}
\maketitle
	
\section{Aufgabenstellung}

Das Ziel der Praktikumsaufgabe ist das Kennenlernen der Bibliothek Graphulo. Dabei handelt es sich um eine Java-Implementierung des GraphBLAS-Standards. Dieser umfasst grundlegende Matrixoperationen, mithilfe derer Entwickler Graphalgorithmen umsetzen können. \cite{graph}

Graphulo nutzt die verteilte Key-Value-Datenbank Accumulo als Datenspeicher. Die Datenbank basiert auf dem Design von Googles BigTable und nutzt Technologien wie das Hadoop File System und Zookeeper. \cite{acc} 

Beispielhaft soll im Rahmen des Praktikums der DBLP-Datensatz, zur Verfügung gestellt von der Universität Trier, untersucht werden. DBLP steht dabei für \glqq Digital Bibliography \& Library Project\grqq und umfasst Informationen über eine Vielzahl von Veröffentlichungen aus dem Bereich der Informatik. \cite{dblp}


\section{Umsetzung}

Die Umsetzung umfasst zwei verschiedene Schwerpunkte. 
%TODO einerseits Top-Down-Analyse, Vergleich...
%TODO aufwand für verteilte accumulo-installation bzw graphulo ausrollung auf solcher

Außerdem ist es ein Ziel, Analysen auf dem DBLP-Datensatz durchzuführen. Der erste Schritt dafür ist, eine lokale Accumulo-Instanz zu installieren und lauffähig zu machen.

Anschließend erweitern wir den \cite{pars} um eine Schnittstelle, die geparste Daten in ein von Graphulo nutzbares Format konvertiert und in einer Accumulo-Instanz speichert.

Graphulo bietet neben Basismatrixoperationen bereits einige beispielhafte Graphalgorithmen an. Zuerst möchten wir versuchen, mithilfe dieser den Datensatz zu analysieren. Weitergehend möchten wir selbst einen Graphalgorithmus auf Basis der GraphBLAS-Operationen implementieren. Hierbei entschieden wir uns für einen Algorithmus zum Finden von Connected Components. 


\section{Geplantes Ergebnis}

%TODO das nichttechnische ergebnis...

Das technische Ergebnis stellt eine einfache Java-Schnittstelle für Accumulo dar, die Daten in einem für Graphulo nutzbaren Format akzeptiert. Außerdem soll eine Implementierung eines Connected-Components-Algorithmus entstehen, der in Graphulo einsetzbar sein soll.



\begin{thebibliography}{laengste Labelbreite}
	\bibitem[Graphulo]{graph} \url{http://graphulo.mit.edu}, 19.05.2017
	\bibitem[Accumulo]{acc} \url{https://accumulo.apache.org}, 19.05.207
	\bibitem[DBLP]{dblp} \url{http://dblp.uni-trier.de}, 19.05.207
	\bibitem[DBLP-Parser]{pars} \url{https://github.com/ScaDS/dblp-parser}, 19.05.207
\end{thebibliography}

\end{document}